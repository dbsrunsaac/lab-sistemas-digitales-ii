\documentclass[stu, 12pt, floatsintext, ]{apa7}

%opening
\title{Memoria ROM}
\author{Davis Bremdow Salazar Roa}

% 
\usepackage[spanish]{babel}
\usepackage{float}
\begin{document}
	
	\maketitle
	\newpage
	
	% Desarrollo
	\section{Introducción}
	En un inicio el almacenamiento de información se realizaba de forma analógica mediante el uso de papel como el método mas confiable para poder registrar información con el paso del tiempo y el desarrollo tecnológico el almacenamiento paso a ser digital mediante el empleo de materiales semiconductores capaces de almacenar un bit como la mínima y más básica unidad de almacenamiento.
	
	Con el nacimiento de este innovador método de almacenamiento con el tiempo se fueron desarrollando dispositivos capaces de almacenar grandes cantidades de información en el rango de los gigabytes, terabytes, etc. Conforme se fueron agregando más dispositivos de almacenamiento.
	
	En la actualidad existen diferentes tipos de memorias según el propósito deseado dentro de estas se encuentra las RAM, ROM, PROM, EPROM las cuales tienen diferentes características siendo la ROM una memoria estática o de solo lectura.
	
	\newpage
	
	\section{Memorias ROM}
	Las memorias ROM son dispositivos del almacenamiento de solo lectura esto implica que la información una vez grabada esta no puede modificarse o se requiere de procesos específicos según el tipo de memoria ROM.
	
	La forma en la que una memoria ROM almacena información es mediante una matriz de dispositivos básicos de información siendo así que estas organizan con una longitud de palabra de 8 bits o 1 byte y una cantidad de direcciones equivalente a una potencia de 2 que fueron el estándar creado para ambas propiedades antes mencionadas.
	
	\section{Familias de las ROM}
	Una memoria ROM se sub-divide en diferentes categorías según el método de reescritura 
	\section{}
	
\end{document}
